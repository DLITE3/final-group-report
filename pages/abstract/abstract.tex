\centerline{
  \huge\textbf{概要}
}
\vspace{1cm}
\noindent
\space 色覚障害者は、健常者とは色が異なって見える。そのため、仕事や日常生活に支障をきたす場合がある。そこでカラーユニバーサルデザインを用いて色覚障害を持った人が認識することが難しい色の組み合わせを認識が容易である色の組み合わせに変換するアプリケーションを開発する。このアプリケーションは色覚障害の種類に応じて認識が難しい色の組み合わせを見やすい色に変換する。色覚障害とは、人間の網膜に存在する明るい場所で働く視細胞(錐体細胞)のうち、赤などの長波長に反応する細胞(L錐体)、緑などの中間波長に反応する細胞(M錐体)、青などの短波長に反応する細胞(S錐体)のうち一つ、もしくは一つ以上の異常によって発症する。また、異常のある錐体ごとに見分けやすい色が異なる。色覚障害は親から子へと遺伝することがあり、遺伝による色覚障がいに対する治療方法は現在でも確立されていない。そこで、本研究ではPythonを用いた色覚障害者に対する色識別補助を目的としたウェブページの開発および公開方法について検討を行った。特に、PythonAnywhereを利用したウェブページの構築と公開プロセスに焦点を当て、画像処理やユーザーインタラクションを含むウェブ機能の実装と、それに伴う技術的選択について論じる。\\

\noindent\textbf{キーワード} \indent 色覚障害, カラーユニバーサルデザイン

\newpage

\centerline{
  \huge\textbf{Abstract}
}
\vspace{1cm}
\noindent
\space Individuals with color vision deficiency perceive colors differently from those with normal vision. This can lead to difficulties in daily life and work. To address this issue, we propose developing an application utilizing Color Universal Design principles to convert color combinations that are challenging for individuals with color vision deficiency into combinations that are easier to recognize.

This application adapts to the specific type of color vision deficiency, modifying color combinations that are hard to distinguish into more discernible ones. Color vision deficiency occurs due to abnormalities in the cone cells of the retina, which function in bright environments. These cone cells include L-cones that respond to long wavelengths such as red, M-cones that respond to medium wavelengths such as green, and S-cones that respond to short wavelengths such as blue. The condition arises from defects in one or more of these cone types, and the distinguishable colors vary depending on the affected cone.

Color vision deficiency can be inherited, and no definitive treatment for hereditary color vision deficiency has been established to date. Therefore, this study explores the development and publication of a web application aimed at assisting individuals with color vision deficiency in color recognition, using Python as the primary programming language. The study focuses on constructing and publishing the web application through PythonAnywhere, implementing features such as image processing and user interaction, and discusses the technical choices made during the development process.\\

\noindent\textbf{\textsf{Keyword}} \indent Color vision deficiency, Color Universal Design