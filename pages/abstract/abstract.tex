\centerline{
  \huge\textbf{概要}
}
\vspace{1cm}
\noindent
\space 本プロジェクトでは、「視覚や聴覚に頼れない状況で役立つ装置の開発」をコンセプトとし、障がい者が抱える問題を当事者目線で
検討し、実用的な装置の開発に取り組んできた。頼れない感覚を別の手段で補うことで、不便を解消し、安全で快適な生活を支援す
ることを目指している。聴覚障がいや視覚障がい、色覚の障がい者を対象とした4つのグループに分かれ、それぞれ、特定の言葉や
音に反応するデバイス、画像の色をユニバーサルデザインに変換するアプリ、自力で避難することが難しい人のための補助デバイ
ス、障がい者が自然を楽しむためのデバイスの開発を行っている。\\

\noindent\textbf{キーワード} \indent 障がい者支援, 聴覚補助, 色覚補助, 自然エンタメ

\newpage

\centerline{
  \huge\textbf{Abstract}
}
\vspace{1cm}
\noindent
\space Under the concept of "developing devices that are useful in situations where one cannot rely on sight or hearing," this project examines the
problems faced by people with disabilities from the perspective of the people concerned, to develop practical devices. By supplementing
unreliable senses with other means, the project aims to eliminate inconvenience and support safe and comfortable living. The project is
divided into four team targeting people with hearing disabilities, visual disabilities, and color blindness. Each team is developing devices that
respond to specific words and sounds, applications that convert the color of images to universal design, assistive devices for people who
have difficulty evacuating on their own, and devices that allow people with disabilities to enjoy nature.\\

\noindent\textbf{\textsf{Keyword}} \indent Disability Assistance, Hearing Assistance, Color Assistance, Natural Entertainment